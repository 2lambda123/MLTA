\chapter{Логика предикатов}
\labch{predicates}

\section{Определения}
\begin{definition}
n-местный предикат --- это отношение на декартовом произведении n множеств. 
\end{definition}
\marginpar{
Если говорить простым языком, предикат это некая функция $A_1\times A_2 \times
\ldots \times A_n \to \mathbb{B}$.}
При этом множества $A_1, A_2, \ldots, A_n$
могут быть различными, но если взять $M=A_1\cup A_2 \cup \ldots \cup A_n$, то
можно говорить о n-местном предикате на $M^n$

Одноместный предикат --- свойство\sidenote{Например, P(x) = x --- чётное число},
двухместный\sidenote{Например, $S(x, y) = x<y$} ---
бинарное отношение, известное из предыдущих курсов.

\begin{definition}
	Пусть $A(x, y_1, \ldots, y_n)$ --- некоторый предикат от $n+1$
	переменной $x, y_1, \ldots, y_n$. Высказывание "{}$A(x, y_1, \ldots,
	y_n)$ истинно для всех $x$"{} (для любого, для каждого) обозначается символом $(\forall x) A(x,
	y_1, \ldots, y_n)$. Квантор $\forall$ называется \textbf{квантором
	всеобщности}, переход от предиката $A(x, y_1, \ldots,$$y_n)$ к  $\left(
\forall x \right) A(x, y_1, $$\ldots, y_n)$ называется навешиванием квантора
всеобщности.
\end{definition}

Выражение $(\forall x) A(x, y_1, \ldots, y_n)$ зависит только от переменных
$y_1, \ldots,y_n$, которые называются свободными, является n местным предикатом.
При этом это выражение будет истинно на произвольном наборе свободных переменных
тогда и только тогда, когда оно истинно для любого значения х.

\begin{definition}
	Пусть $A(x, y_1, \ldots, y_n)$ --- некоторый предикат от $n+1$
	переменной $x, y_1, \ldots, y_n$. Высказывание "{}$A(x, y_1, \ldots,
	y_n)$ истинно при некотором $x$"{} обозначается символом $(\exists x) A(x,
	y_1, \ldots, y_n)$. Квантор $\exists$ называется \textbf{квантором
	существования}, переход от предиката $A(x, y_1, \ldots,$$ y_n)$ к  $\left(
\exists x \right) A(x, y_1,$$\ldots,$$ y_n)$ называется навешиванием квантора
существования.
\end{definition}
\marginpar[3cm]{Существует x для которого $A(x, y_1, \ldots, y_n)$}	
Выражение  $\left( \exists x \right) A(x, y_1,\ldots, y_n)$ является n-местным
предикатом, переменная x называется связанной, выражение истинно для данного
набора свободных переменных тогда и только
тогда, когда существует хотя бы одно значение связанной переменной, при котором
оно истинно.
\marginpar{Связанная переменная x и переменная х, которая будет использована в
другой части формулы это разные буквы, например, в $P(x) \lor (\forall x) Q(x,
y)$ х в Р(х) и х рядом с квантором --- это разные х.}

Буквы начала латинского алфавита (в том числе с числовыми индексами) ---
\textbf{предметные константы}. Буквы конца латинского алфавита (в том числе с числовыми
индексами) --- \textbf{предметные переменные}. Буквы $A_i^n$ с числовыми индексами $i>1,
n>0$ называются \textbf{предикатными буквами}. Буквы $f_i^n$ ---
\textbf{функциональные буквы}\marginpar{функциональные буквы обозначают обычные
функции}. Верхний индекс предикатной и функциональной буквы
обозначает число переменных, нижний индекс служит для различения букв. 

Функциональные буквы, применённые к предметным переменным и константам порождают
термы. 
\begin{definition}
	Определение терма:
	\begin{enumerate}
		\item всякая предметная переменная или предметная постоянная
			есть терм;
		\item если $f_i^n$ функциональная буква и $t_1, t_2, \ldots,
			t_n$ --- термы, то $f_i^n(t_1, \ldots, t_n)$ есть терм;
		\item выражение является термом только в том случае, если это
			следует из двух условий выше (другими словами, других
			термов нет).
	\end{enumerate}
\end{definition}
 \begin{definition}
	 Если $A_i^n$ --- предикатная буква, а $t_1,\ldots, t_n$ --- термы, то
	 $A_i^n(t_1, \ldots, t_n)$ --- элементарная формула.
\end{definition}
\begin{definition}
	Определение формулы:
	\begin{enumerate}
		\item всякая элементарная формула есть формула;
		\item если $\mathcal{A}$ и $\mathcal{B}$ --- формулы и $y$ ---
			предметная переменная, то каждое из выражений $\left(
			\lnot \mathcal{A} \right) $, $\left( \mathcal{A} \land
		\mathcal{B} \right) $, $(\mathcal{A} \lor \mathcal{B})$,
		$\forall y \mathcal{A}$ есть формула;
	\item выражение является формулой только в том случае, если это следует
		из двух предущих пунктов.
	\end{enumerate}
\end{definition}
В пункте 2 использовалась система логических связок $\left\{ \lor, \land, \lnot
\right\} $, однако также может быть использована любая другая полная систему
функций.

Областью действия квантора $\forall y$ называется формула $\mathcal{A}$ в
выражении  $\forall y \mathcal{A}$.

Кванторы по старшинству операций располагаются после  $\lnot, \land, \lor$. 

$\exists x \mathcal{A} = \lnot\left( \forall x (\lnot \mathcal{A} )\right) $ 

\begin{definition}
	Длина l(F) формулы F --- это количество входящих в её запись букв и
	логических связок.
\end{definition}

\begin{definition}
	Вхождение переменной $x$ в данную формулу называется связанным, если
	$x$ является переменной входящего в эту формулу квантора $\forall x$, в
	противном случае вхождение --- свободное. 
\end{definition}

\begin{definition}
	Формула называется замкнутой, если она не содержит никаких свободных
	переменных.
\end{definition}

\section{Интерпретации. Выполнимость и истинность. Модели}
\marginpar{Например, зададим интерпретацию $\forall x A(x, y, b)$. Пусть M ---
множество целых чисел, b = 5, а предикатная буква A выражает предикат
"х не меньше y + b"{}.}
\begin{definition}
Под интерпретацией понимается совокупность:
\begin{enumerate}
	\item заданного непустого множества M, называемого областью
		интерпретации;
	\item заданного соответствия:
		\begin{itemize}
			\item Предикатным буквам $A^n$ --- некоторые n местные
				предикаты в M.
			\item Функциональным буквам $f^n$ --- некоторые n
				аргументные функции (операции) в M, отображающие
				$M^n  \to  M$.
			\item Каждой предметной постоянной --- некоторые предмет
				из M.
			\item Кванторам и связкам --- их обычных смысл.
		\end{itemize}
		
\end{enumerate}
\end{definition}

Замкнутая формула представляет собой высказывание, которое может быть либо
истинно, либо ложно, а формула со свободными переменными выражает некоторое
отношение на области интерпретации.

При этом для любой формулы можно построить бесчисленное множество разных
интерпретаций.

\begin{definition}
	Формула называется выполнимой в данной интерпретации, если она принимает
	значение Истина хотя бы для одной совокупности возможных значений
	свободных переменных (если они есть). Если формула не содержит свободных
	переменных, то она называется выполнимой в том случае, если принимает
	значение Истина в этой интерпретации.
\end{definition}

\begin{definition}
	Формула называется истинной в данной интерпретации, если она принимает
	значение Истина для всех возможных значений её свободных переменных,
	если они есть. Если же свободных переменных нет, то формула называется
	истинной, когда она принимает значение Истина в этой интерпретации.
\end{definition}

\begin{definition}
	Формула называется ложной в данной интерпретации, если она принимает
	значение Ложь для всех возможных значений её свободных переменных (если
	они есть). Если же свободных переменных нет, то формула называется
	ложной, когда она принимает значение Ложь в этой интерпретации.
\end{definition}

\begin{definition}
Интерпретация называется моделью для данного множества формул $\Gamma$, если
$\forall \mathcal{A} \in \Gamma$, $\mathcal{A}$ истинна в данной интерпретации.
\end{definition}

Свойства\sidenote{больше свойств приведено на страницах 59-62 Мендельсона.} формул в заданной интерпретации:
\begin{enumerate}
\item А ложна в данной интерпретации $\leftrightarrow$ $\lnot A$  истинна в этой
	же интерпретации. И наоборот.
\item Никакая формула не может быть одновременно истинной и ложной в данной
	 интерпретации
\item Если в данной интерпретации А --- истинна и $A\to B$, то B тоже истинна
\item Формула $A\lor B$ выполнима в данной интерпретации, если хотя бы одна из
	них выполнима в данной интерпретации
\item Формула $\exists x A$ выполнима в данной интерпретации  $\leftrightarrow$
	А принимает значение Истина хотя бы для одной совокупности значений её
	свободных переменных и хотя бы одного значения переменной х.
\item Формула $\forall xA$ истинна в данной интерпретации  $\leftrightarrow$ в
	этой интерпретации истинна А.
\end{enumerate}

\begin{definition}
	Формула логики предикатов называется логически общезначимой, если она
	истинна в любой интерпретации. 
\end{definition}

Например, $\forall xA \to \exists xA$.
\begin{definition}
	Формула логики предикатов А называется выполнимой, если существует
	интерпретация, в которой выполнима А.
\end{definition}

\begin{proposition}
Формула А логически общезначима тогда и только тогда, когда $\lnot A$ не является
выполнимой.
\end{proposition}

\begin{proposition}
	Формула А выполнима тогда и только тогда, когда $\lnot A$ не является
	логически общезначимой.
\end{proposition}

\begin{definition}
	Формула логики предикатов $\mathcal{A}$ называется противоречием, если
	формула $\lnot \mathcal{A}$ является логически общезначимой. (Формула А
	ложна во всякой интерпретации.)
\end{definition}

\begin{definition}
	Формула логики предикатов А логически влечёт B, если в любой
	интерпретации формула В выполнена на всякой последовательности, на
	которой выполнена А. (В является логическим следствием А).
\end{definition}

\begin{definition}
	Формулы логики предикатов А и В называются равносильными
	(эквивалентными), если каждая из них логически влечёт другую.
	Обозначается $A \sim B$.
\end{definition}

\begin{theorem}
	Формула А логически влечет формулу В тогда и только тогда, когда формула
	$A\to B$ логически общезначима.
\end{theorem}

\begin{theorem}
	Формулы A и B равносильны (логически эквивалентны) тогда и только тогда,
	когда формула $A \sim B$ логически общезначима.
\end{theorem}
\begin{theorem}
	Если формула А логически влечёт формулу В и А истинно в данной
	интерпретации, то в этой же интерпретации истинно и В.
\end{theorem}
\begin{definition}
	Формула В называется логическим следствием формул $A_1, A_2, \ldots,
	A_n$, если в любой интерпретации формула B принимает значение Истина при
	каждой совокупности значений свободных переменных (входящих в В и $A_1,
	\ldots, A_n$ при которых одновременно все формулы $A_1, \ldots, A_n$
	приняли значение Истина. 
\end{definition}

Если формула записана в нормальной форме, то значит, что она записана в
выбранном нами базисе, который мы называем нормальным базисом.

\begin{theorem}
	Правило переименования связанных переменных. В любой формуле связанные
	переменные могут быть заменены на другие переменные (в кванторе и всюду
	в области действия квантора),так, чтобы не нарушалось определение
	формулы, в таком случае мы получим формулу эквивалентную исходной.
\end{theorem}
\begin{proof}
	Пусть дана формула $F$. Пусть в этой формуле встречаются переменные
	(литералы) $x_1, x_2, x_1, x_2, x_1, x_5, \ldots$ (связанные переменные
	будем записывать только один раз, то есть из формулы $\forall x_1 P(x_1)
	\lor Q(x_1)$ мы выпишем $x_1, x_1$, первая - связанная, вторая свободная). Отметим там 
	все вхождения литералов, обозначающие связанные переменные. Пусть мы
	отметили k литералов. Теперь составим множество всех литералов $x_1,
	x_2, \ldots, x_s$, и добавим к нему литералы $x_{s+1}, \ldots, x_{s+k}$.
	Проведём биекцию: первой отмеченной сопоставим $x_{s+1}$, второй ---
	$x_{s+2}$, и т. д. Теперь пере обозначим все отмеченные буквы в формуле
	на те, которые мы поставили им в соответствие, учитывая, что, когда мы
	переименовываем связанную переменную, нужно переименовать и все её
	вхождения в её области действия.
\end{proof}

\begin{definition}
	Формула исчисления предикатов следующего вида
	$F=Q_1x_1Q_2x_2\ldots Q_nx_n G$, где $Q_1, \ldots, Q_n$ --- кванторы, а
	формула G не содержит кванторов, называется формулой в предварённой
	(приведённой) нормальной форме.
\end{definition}

\begin{theorem}[без доказательства]
	Для каждой формулы логики предикатов в нормальной форме со связками
	$\{\lnot, \to\}$ существует логически эквивалентная формула в нормальной
	форме со связками $\{\lnot, \lor, \land\}$, имеющая ту же длину
\end{theorem}

\begin{theorem}
	Для любой формулы логики предикатов существует эквивалентная ей формула
	в предварённой нормальной форме той же длины.
\end{theorem}

\begin{theorem}
	Пусть F --- формула длины l(F), $F_1$ --- эквивалентная ей формула в
	приведённой нормальной форме длины $l(F_1)$, тогда $l(F) = C \cdot
	l(F_1)$.
\end{theorem}

\section{Класс P/Poly}

\begin{definition}
	$P_{Poly} = \cup\limits_{c\in \mathbb{N}} SIZE(n^c) $
\end{definition}
$SIZE(n^c)$ --- множество языков, разрешимых СФЭ полиномиальной сложности.

По сути класс $P_{Poly}$--- семейство всех таких языков $L \subset \left\{ 0,1
\right\}^* $, схемная сложность которых есть функция полиномиальной сложности.

Ключевой вопрос --- как наращивать сложность, чтобы оценить, полиномиальная она
или нет. Рассмотрим предикат, определённый на множестве двоичных слов, TBA


\begin{theorem}
	\labthm{PinPPoly}
	$P\subseteq P_{Poly}$
\end{theorem}
\begin{proof}
	Пусть есть задача $Z \in P$, нужно показать, что эта задача будет лежать
	и в  $P_{Poly}$.

	Для любой индивидуальной задачи I существует таблица выполнимости,
	которая состоит из  $n^k$ строк (так как задача в P, то она разрешима на
	МТ за полином). Данной таблице выполнимости поставлена в соответствие
	КНФ $\phi = \phi_{start}\cdot \phi_0 \cdot \phi_{accept} \cdot
	\phi_{compute}$, длины $O(n^{2k})$ (см. доказательство Теоремы Кука).
	Значит, СФЭ этой функции также будет
	$O(n^{2k})$. Получается, для любой индивидуальной задачи можно построить
	СФЭ полиномиальной сложности, значит $Z\in P_{Poly}$.
\end{proof}
\begin{theorem}
	$P \not = P_{Poly}$
\end{theorem}
\begin{proof}
	Пусть $\phi(x): \mathbb{N} \to {0, 1}$ --- алгоритмически неразрешимая
	задача. Мы знаем, что предикат принимает значения либо 0, либо 1. Тогда
	рассмотрим предикат $P(x_1, x_2,\ldots, x_n) = \phi(\left| x_1, x_2,
	\ldots, x_n \right| )$. Данный предикат не может быть алгоритмически
	разрешим, так как не существует Машины Тьюринга, которая могла бы
	вычислить значение $\phi$. А значит он не лежит в $\mathcal{P}$. При этом  $\forall n P(x_1, \ldots, x_n) =
	const = \phi (n), \forall x_1, \ldots, x_n$. Так как сложность
	реализации СФЭ константы всегда константа, то такой предикат P лежит в
	$P_{Poly}$.
\end{proof}

\begin{proposition}
	Чтобы для $Z\in P_{Poly} \to Z \in P$, все $f_1(x_1), \ldots, f_n(x_1,
	\ldots, x_n)$ должны ограничиваться полиномами и для этой $Z$ должен
	существовать алгоритм решения на Машине Тьюринга. 
\end{proposition}


\begin{theorem}
	Задача $Z $ принадлежит классу P тогда и только тогда, когда выполнены
	два условия 
	\begin{enumerate}
		\item $Z\in P_{Poly}$ 
		\item Для $\forall f_n(x_1, \ldots, x_n)$ можно построить СФЭ
			полиномиальной сложности.
	\end{enumerate}
\end{theorem}
\begin{proof}
	$\left( \implies \right) $. Если $Z\in P$, то по \refthm{PinPPoly}
	выполняются оба условия.
$\left( \impliedby \right) $ Так как $Z\in P_{Poly}$ мы за полином можем вычислить
	$F_Z$, то $Z\in P$
\end{proof}


