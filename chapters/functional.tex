\chapter{Схемы из Функциональных Элементов}
\labch{functional}
\section{Определения}

\begin{definition}
	Дерево --- связный граф без циклов.
\end{definition}

\begin{definition}
	Корневое дерево --- это дерево, у которого помечена одна вершина.
\end{definition}

\begin{definition}
	Помеченное дерево --- дерево, все вершины которого помечены
	(пронумерованы).
\end{definition}

При разговоре о изоморфизме, стоит понимать, что два дерева могут быть
изоморфны, но при этом их помеченные версии равно как и корневые могут быть не
изоморфны.

Также в этом параграфе приведём утверждения, известные из курса Теории Графов.

\begin{proposition}
В любом конечном связном графе существует подграф, содержащий все вершины
исходного графа и являющийся деревом.	
\end{proposition}
\begin{proof}
Из каждого цикла будем удалять по одному ребру, пока циклов не останется. Так
как мы разрезаем циклы, то связность от этого рушиться не будет. Так как циклов
конечное число, процесс когда-нибудь остановится.
\end{proof}
\begin{corollary}
Всякий связный граф можно получить из дерева достройкой рёбер.
\end{corollary}
\begin{proposition}
	В дереве из n вершин $(n-1)$ ребро.
\end{proposition}
\begin{definition}
	Граф называется ориентированным, если каждому его ребру приписано
	направление. Также такой граф сокращённо называется орграфом.
	
	Для каждого ребра одну вершину будем называть началом, другую ---
	концом. Направление ребра идёт от начала к концу.
\end{definition}
\begin{definition}
	Ориентированный цикл --- цикл в ориентированном графе, в котором все
	рёбра направлены в одну сторону, т.е. конечная последовательность
	ориентированных рёбер, где конечная вершина совпадает с начальной.
\end{definition}
\begin{definition}
	Степень захода вершины в орграфе --- количество рёбер, конец которых
	находится в данной вершине (направленных в данную вершину).

	Степень исхода вершины в орграфе --- количество рёбер, начало которых
	находится в данной вершине.
\end{definition}
\begin{lemma}
\lablemma{orgraph-begining}
В орграфе без ориентированных циклов существует вершина со степенью исхода 0. 
\end{lemma}
\begin{proof}
Докажем от противного. Допустим, таких вершин не существует, то есть каждая
вершина обязательно является началом для какого-либо ребра. Возьмём случайную
вершину и начнём идти по рёбрам. В какой-то момент мы упрёмся в вершину, в
которой мы уже были (вершин конечное число, а значит бесконечно в новые мы идти
не сможем), получается, в графе есть орцикл. Противоречие.
\end{proof}

\begin{theorem}
\labthm{orgraph-numeration}
В любом конечном орграфе из n вершин без ориентированных циклов можно занумеровать вершины
первыми n натуральными числами так, чтобы для каждого ребра номер конца был
больше, чем номер начала.
\end{theorem}
\begin{proof}
Докажем по индукции.

База. Граф из одной вершины можно пометить очевидным образом. 

Пусть нам нужно пометить орграф из n вершин, но мы знаем, что пометить можно любой
орграф без орциклов из n-1 вершины. Выберем в графе из n вершин такую, у которой
степень исхода 0 (\reflemma{orgraph-begining}). Удалим её из графа, получим
помеченный граф из n-1 вершины, удалённую вершину пометим как n.
\end{proof}
\begin{definition}
	\textbf{Схема из функциональных элементов} (СФЭ) --- конечный
	ориентированный граф без ориентированных циклов, все вершины которого
	имеют степени захода не более 2 и степени исхода не более 1. Вершинам
	степенью захода приписывается символ-переменная $x_i$, вершины со
	степенью захода 1 получают символ $\neg$, вершины со степенью захода 2
	помечаются $\land$ или $\lor$. Некоторым вершинам приписывается $*$.

	Занумеруем вершины графа согласно \refthm{orgraph-numeration}. Каждой
	вершине СФЭ можно сопоставить некоторую булеву функцию по следующему
	правилу. Пусть всем вершинам с номерами $1,\ldots, n-1$ уже
	соответствуют булевы функции. Если у $n$-й вершины степень захода 0, то
	ей приписана переменная, которую мы и поставим ей в соответствие. Если у
	n-й вершины степень захода 1, то
	ей соответствует отрицание функции той вершины, из которой исходит ребро.
	Если у n-й вершины степень захода 2, то ей будет соответствовать
	конъюнкция или дизъюнкция функций вершин, из которых исходят рёбра. По
	правилу нумерации из \refthm{orgraph-numeration} у вершин, из которых
	исходят рёбра номера меньше, чем у вершин, в которых рёбра входят,
	значит функции для них уже известны.
\end{definition}

\section{Оценки}
\marginnote{Сейчас может быть непонятно, зачем мы приводим некоторые оценки, но
они понадобятся в \refsec{Shennon}.}

\begin{proposition}
	\labprop{upper-bound-c-n-k}
	Имеет место оценка $C_n^k \le 2^n$.
\end{proposition}
\marginnote{
В этом разделе речь идёт о связных графах.}
\begin{theorem}
\labthm{treerootup}
	Пусть $\delta^*(q)$ количество неизоморфных корневых деревьев c q рёбрами. Тогда имеет место оценка \[
		\delta^*\left( q \right) \le 4^q
.\] 
\end{theorem}
\begin{proof}
Пусть дано корневое дерево. Расположим его по ярусам так, чтобы корень находился
в самом низу, а концы всех рёбер находились на соседних ярусах, и зафиксируем
эту расстановку. Начинаем ходить по дереву по следующему алгоритму:
\begin{enumerate}
	\item Если из данной вершины есть ребро с концом в вершине ярусом выше,
		в которой мы ещё не были,
		то поднимаемся в эту вершину. Если таких ребёр несколько, то
		выбираем ребро с самой левой вершиной среди тех, в которых мы
		ещё не были.
	\item Если для данной вершины не осталось рёбер, по которым мы не
		ходили, спускаемся вниз по ребру, которое соединяет эту вершину
		и вершину на более низком ярусе.
	\item Если мы обошли все рёбра сверху и спуститься не можем, значит мы
		вернулись в корень и можем остановиться.		
\end{enumerate}
Каждому нашему подъёму поставим в соответствие последовательность из 0 и 1. 1
будем писать тогда, когда мы поднимаемся на ярус выше, 0, когда спускаемся.
Так как по каждому ребру мы пройдём два раза, то у нас будет записано $2q$ цифр.
Количество способов записать последовательность последовательность из $2q$ нулей
и единиц --- $2^{2q} = 4^q$. Это число не меньше, чем число корневых деревьев.
Теорема доказана.
\end{proof}
\begin{theorem}
	\labthm{treeup}
	Количество неизоморфных деревьев с q рёбрами $\delta(q)\le 4^q$.
\end{theorem}
\begin{proof}
Очевидно, что количество неизоморфных корневых деревьев больше, чем число
неизоморфных деревьев, так как из любого дерева можно получить корневое выбрав
одну из n вершин в качестве корня. Следовательно, из \refthm{treerootup}
получаем, что $\delta(q)\le\delta^*(q)\le 4^q = 4^{n-1}$
\end{proof}
\begin{theorem}
Количество неизоморфных деревьев с q рёбрами оценивается снизу: \[
	\delta(q) \ge \frac{1}{2 \sqrt[3]{2}}\left( \sqrt[3]{2}  \right) ^q
.\] 
\end{theorem}
\begin{proof}
Рассмотрим деревья только определённого вида. Понятно, что количество таких
неизоморфных деревьев будет оценкой снизу для общего числа неизоморфных
деревьев.

Вид деревьев представлен на рисунке \ref{fig:trees}
\begin{figure*}[h!]
    \centering
    \incfig{trees}
    \caption{Вид дерева и присоединяемых поддеревьев.}
    \label{fig:trees}
\end{figure*}

На каждую из вершин $1,\ldots, t$ добавляется один из подграфов, причём
совмешается помеченная и обведённая в кружок вершины. 

Посчитаем число рёбер в этом графе. $q=4+3t$. 4 ребра исходят из 0, а из остальных
t вершин можно считать, что исходит ребро, соединяющее  ${t-1, t}$ и два ребра
из присоединённого подграфа.

Понятно, что к каждой из t вершин можно присоединить один из двух подграфов,
получается, мы получим $2^t$ неизоморфных дерева. 

Из того, что $t = \frac{q-4}{3}$, выражаем число неизоморфных деревьев через число рёбер: \[
	2^t = 2^{\frac{q-4}{3}} = \frac{\left( \sqrt[3]{2}  \right)^q
	}{2\sqrt[3]{2} } \le \delta(q)
.\] 
\end{proof}
\begin{theorem}
	Число неизоморфных графов с n вершинами и m рёбрами \[\gamma(n, m) \le n^{m-n} \cdot
	A^{n+m},\] $A=const$
\end{theorem}
\begin{proof}
Число неизоморфных графов можно оценить сверху как произведение числа
неизоморфных деревьев на n вершинах (это уже n-1 ребро, \refthm{treeup}	) и числа способов
построить необходимое $m-n+1$ ребро. Количество способов построить k рёбер
можно оценить сверху как $\overline{C}_n^k \cdot  n\cdot k$ (выбираем один конец
ребра и для него выбираем второй конец ребра).
Получаем формулу \[
	\gamma(n, m) \le \delta(n-1) \cdot \overline{C}_n^{m-n+1} n^{m-n+1} \le
	4^{n-1}\cdot C_{m+1}^{n}\cdot n^{m-n+1}\le\] 
	\[
		\le 4^{n-1} \cdot 2^{m+1} n n^{m-n}\le4^{n-1} \cdot 2^{m+1} 2^n
		n^{m-n} \le 8^n 8^m n^{m-n} \le A^{n+m} n^{n-m}
	.\] 
\end{proof}
\section{Функция Шеннона}
\labsec{Shennon}

Каждоый булевой функции можно поставить в соответствие какой-то функциональный
элемент. Если функция от k переменных, то функциональный элемент будет иметь k
входов и 1 выход.

\marginnote{
\begin{theorem}
	Теорема Поста.\\
Система булевых функций F является полной тогда и только тогда, когда она
целиком не принадлежит ни одному из замкнутых классов $T_0, T_1, S, M, L$.
\end{theorem}
}
Базисом будем называть какое-то множество булевых функций.
\begin{definition}
	Базис называется полным, если множество функций, которые он содержит
	полное по Посту.
\end{definition}

Если базис полный, то любую функцию можно реализовать в виде формулы в этом
базисе. 

Будем говорить, что СФЭ $S_f$ реализовывает булеву фукнцию f.\sidenote{Одну и ту
же фуннкцию даже в одинаковом базисе могут реализовывать разные СФЭ.}

\begin{definintion}
	Сложностью схемы из функциональных элементов называется количнство
	элементов в этой схеме.
\end{definintion}

$L_B(f)$ --- сложность минимальной схемы для f в базисе B.

$L_B(n)$ --- функция Шеннона --- количество элементов в схеме, которая
реализовывает любую функцию от n в базисе B.  $L_B(n) = \max\limits_{f\in
B_n}L_B(f) $.

\begin{theorem}
	Если $B_1, B_2$ --- два полных базиса, то $L_{B_1}(n)$ и  $L_{B_2}(n)$
	отличаются не более чем в константу.
	
	Иными словами \[
		L_{B_1}(n) \le c_1 L_{B_2}(n), L_{B_2}(n) \le c_2 L_{B_1}(n)
	.\]
\end{theorem}
\marginnote[-1cm]{без доказательства}
\marginnote{Далее в конспекте будет идти речь о стандартном базисе $\{\neg, \land, \lor\}$ и
буква B будет опускаться.}
\begin{theorem}
	Справедливо неравенство $L(n) \le (n+1)2^{n-1} \le n2^n$
\end{theorem}
\begin{proof}
Рассмотрим СДНФ булевой функции. По своей форме СДНФ --- дизъюнкция мономов,
которые являются конъюнкциями n переменных, каждая из которой может быть в
степени 1 или 0. Таких мономов не более $2^n$, в каждом мономе $(n-1)$ конъюнкция.
C отрицаниями поступим просто: для каждой переменной сразу получим её отрицание
и там, где оно будет нужно, будем использовать его, таким образом мы используем
ещё n функциональных элементов. Получается, каждую
булеву функцию можно реализовать не более чем  $\left( n-1 \right) 2^n + n$
элементами. 

Теперь рассмотрим СКНФ булевой функции. При нахождении сложности реализации
любой СКНФ можно провести аналогию с СДНФ. На место мономов из n-1 конъюнкции
приходит полином из n-1 дизъюнкции, а эти полиномы между собой объединены
конъюнкцией. Получается, любая функция реализовывается через СКНФ также не более
чем $\left( n-1 \right) 2^n + n$ элементами.

Количество конъюнкций в СКНФ зависит от числа нулей в векторе значений, а
количество дизъюнкций в СДНФ зависит от числа единиц в векторе значений. Тогда
давайте брать СКНФ, если нулей меньше, чем единиц, а СДНФ во всех остальных
случаях. Получается, вместо $2^n$ мы можем взять $2^{n-1}$. Получается, каждая
функция может быть реализована не более чем $2^{n-1}\left( n-1 \right) + n \le
2^{n-1}(n+1)$.

Неравенство $2^{n-1}(n+1)\le_2^n n$ я считаю очевидным.
\end{proof}
Дальнейшие оценки будут ассимптотическими.
\begin{theorem}
\labthm{universal-conunctor}
Сложность универсального конъюнктора от n переменных (СФЭ, которая принимает на
вход n переменных, а на выходе имеет все возможные конъюнкции от n переменных)
$C(n) \sim 2^n.$
\end{theorem}
\begin{proof}
Пусть сложность универсального конъюнктора от n переменных обозначается $k_n$.
Давайте выразим $k_n$ через $k_{\frac{n}{2}}$. Конъюнкций от $n$ переменных --- $2^n$.
Возьмём два универсальных конъюнктора: один от переменных $x_1,x_2,
x_{\frac{n}{2}}$, другой от оставшихся переменных. Сложность каждого из них
будет $k_{\frac{n}{2}}$, а теперь все $2^{\frac{n}{2}}$ выходов первого нужно
попарно проконъюнктировать с выходами второго. Получается, нам потребуется $2^n
+ 2k_{\frac{n}{2}}$ функциональных элементов для этого. \[
	k_n = 2^n + 2\cdot 2^{\frac{n}{2}} + 2^2\cdot 2^{\frac{n}{2^2}} + \ldots
\sim 2^n .\]
\end{proof}
\begin{theorem}
	Справедливо равенство \[
		L(n) \lesssim 12 \cdot \frac{2^n}{n}
	.\] 
\end{theorem}
\begin{proof}
Будем реализовывать формулу вида \[
	f(x_1, x_2,\ldots,x_n) = \bigcup\limits_{\sigma_1,\ldots,
	\sigma_k}x_1^{\sigma_1}x_2^{\sigma_2}\cdot\ldots\cdot x_k^{\sigma_k}
	f(\sigma_1, \ldots, \sigma_k, x_{k+1}, \ldots, x_n)
.\] 
Формулу от $q = n - k$ переменных мы гарантированно реализуем за  $q2^q$
элементов.
\refthm{universal-conunctor} говорит о том, что все конъюнкции от k переменных
можно ассимптотически реализовать за $2^k$ элементов. Ещё нам понадобится
$2^k-1$ дизъюнктор и $2^k$ конъюнкторов (для склеивания f с конъюнкциями).
 \[
	 L(n) \lesssim 2^k - 1 + 2^k + 2^k+ 2^{2^q}\cdot 2^q\cdot q
.\] 
Выберем $q = \left[\log_2 n\right] - 1$, получим оценку \[
L(n) \lesssim 3\cdot 2^k + (\left[\log_2 n\right] - 1 ) \frac{n}{2} \cdot
2^{\frac{n}{2}} = 3\cdot 2^{n-\left[\log_2n\right] +1} + \log_2n \cdot \frac{n}{2}
2^{\frac{n}{2}} = \] 
\[
	= \frac{ 3\cdot 2\cdot 2^n }{2^{\left[ \log_2n \right] }} +  \log_2n \cdot \frac{n}{2}
	2^{\frac{n}{2}} \lesssim \frac{3\cdot 2\cdot 2\cdot 2^n}{n} +  \log_2n \cdot \frac{n}{2}
	2^{\frac{n}{2}} \lesssim 12\cdot \frac{2^n}{n}
.\] 
\end{proof}

\begin{theorem}[О. Б. Лупанов]
	Справедливо неравенство \[L(n) \lesssim \frac{2^n}{n}.\]
\end{theorem}

\begin{definition}
	Пусть $M(n)$ --- количество объектов, а $M_s(n)$ --- количество
	объектов, обладающих каким-то свойтсвом s. Будем говорить, что почти все
	объекты из $M(n)$ обладают свойством s, если \[
		\lim_{n \to \infty} \frac{M_s(n)}{M(n)} = 1
	.\] 
\end{definition}

\begin{theorem}
	Справедливо неравенство \[
		L(n) \gtrsim \frac{2^n}{n}
	.\] 
\end{theorem}
\begin{proof}
Введём следующие обозначения:
\begin{itemize}
	\item $P_2^*(n)$ --- количество булевых функций существенно зависящих от n
		переменных
	\item $N(n, h)$ --- количество функций существенно зависимых от n
		переменных и реализуемых СФЭ не более чем за h элементов.
	\item $N'(n, h)$ --- количество функций существенно зависимых от n
		переменных и реализуемых СФЭ ровно за h элементов.
	\item $N''(n, h)$ --- количество СФЭ сложности h для функций существенно
		зависимых от n переменных.
\end{itemize}
Всего от n переменных $2^{2^n}$ функций. Число зависимых от n переменных булевых
функций можно оценить следующим образом: $|P_2^*(n)| \gtrsim 2^{2^n} - n
2^{2^{n-1}}$, понятно, что $n 2^{2^{n-1}} = o(2^{2^n})$, а значит \[|P_2^*(n)|
\sim 2^{2^n}.\] 

Очевидно, что $N = N'$ (можно дополнить схему ничего не делающими элементами).
Также очевидно, что $N \le N''$. 

Покажем, что  $\forall \epsilon >0, h_0:=\left( 1-\epsilon \right) \frac{2^n}{n}$  \[
	\frac{N''(n,h_0)}{P_2^*(n)} \to\limits_{n \to \infty} 0
,\] то есть почти все функции от n существенных переменных не могут быть
реализованы за $h_0$. Получается, по определению функции Шеннона $h_0 \le L(n)$,
а значит $L(n) \ge \frac{2^n}{n}\left( 1-\epsilon \right) $, $L(n)\gtrsim
\frac{2^n}{n}$.

Величину N будем оценивать, мажорируя её N''. Для этого нам нужно посчитать
число графов, которые содержат $(n+h)$ вершин (n входов и h ФЭ) и q рёбер.
$N(n,h) = \sum_{q=1}^{2h} N''(n, h, q).$ Составляющие суммы будут складываться
из:
\begin{itemize}
	\item $\gamma(h+n, q) = A^{h+n+q}\cdot (h+n)^{q-h-n}$ --- числа неизоморфных графов с h+n вершинами и q
		рёбрами
	\item  $(h+n)^n$ --- оценка сверху количества способов выбрать вход СФЭ
	\item (h+n) --- число способов выбрать выход
	\item $2^q$ --- выбор ориентации рёбер
	\item $3^h$ --- назначение СФЭ ( $\lnot, \land, \lor$)
\end{itemize}
\[
	N''(n, h, q) \le A^{h+n+q} (h+n)^{q-h-n} (h+n)^{n+1} 2^q 3^h\le
	B^{h+n+q} (h+n)^{q-h+1}
.\]
$q \le 2h$
 \[
	 N''(n, h, q) \le B^{3h+n} (h+n)^{h+1}
.\] 
Теперь получим оценку для $N''(n, h)$, учитывая, что меньше, чем h рёбер быть не
может.
 \[
	 N''(n, h) \le \sum_{q=h}^{2h} B^{3h+n} (h+n)^{h+1} =
	 B^{3h+n}(h+1)(h+n)^{h+1}\le (C(h+n))^{h+n}
.\]
Докажем, что $\frac{\left( C(h+n) \right)^{h+n} }{2^{2^n}} \to 0$ при $h_0 =
(1-\epsilon) \frac{2^n}{n}$ \[
	\log_2\left(  \frac{\left( C(h+n) \right)^{h+n} }{2^{2^n}} \right) =
\left( h+n \right) \left[\log_2C + \log_2(h+n)\right] - 2^n \lesssim \] \[\lesssim
	(1-\epsilon)\frac{2^n}{n} \left[ \log_2\left( (1-\epsilon) \frac{2^n}{n}
	+ n\right)  \right] - 2^n \lesssim \] \[\lesssim (1-\epsilon) \frac{2^n}{n} \left[
n - \log_2n \right] - 2^n \lesssim (1-\epsilon)2^n - 2^n = -\epsilon 2^n \to
-\infty
.\]
Очевидно, что если $\log_2x \to -\infty$, то $x\to 0$.

\end{proof}
