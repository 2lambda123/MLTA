\chapter{Классы PSPACE и NPSPACE}

\section{Опеределение}
В рамках рассмотрения классов PSPACE и NPSPACE, мы говорим о трёхленточной
Машине Тьюринга, состоящей из входной, рабочей и выходной лент. Вся запись
вспомогательной информации осуществляется только на рабочую ленту.
\begin{definition}
	Классом PSPACE называется класс языков, допускаемых детерминированной
	Машиной Тьюринга, с длиной рабочей и выходной ленты, ограниченной полиномом от
	длины входной ленты. 
\end{definition}

\begin{definition}
	Классом NPSPACE называется класс языков, допускаемых недетерминированной
	Машиной Тьюринга, с длиной рабочей и выходной ленты, ограниченной полиномом от
	длины входной ленты. 
\end{definition}

\begin{definition}
	Классом $TIME(f(n))$ называется класс языков, допускаемых
	детерменированной машиной Тьюринга со сложностью (за время) ограниченной
	$f(n)$ от длины входа n.
\end{definition}

\begin{definition}
	Классом $SPACE(f(n))$ называется класс языков, допускаемых
	детерминирванной машиной Тьюринга с использованной памятью, ограниченной
	$f(n)$ от длины выхода n.
\end{definition}
\[
	TIME(p(n) = P
.\] 
\[
	TIME(2^{p(n)}) = EXPTIME
.\] 
\[
	SPACE(p(n)) = PSPACE
.\] 

\section{Отношения классов}
\begin{proposition}
	$P \subseteq PSPACE$
\end{proposition}
\begin{proof}
	Если детерменированная машина Тьюринга работает за полином, то она
	сможет прочитать только полиномиальное число ячеек памяти.
\end{proof}

\begin{proposition}
	$NP \subseteq PSPACE$
\end{proposition}
\begin{proof}
	При доказательстве теоремы о том, что задачу, решаемую НМТ за $p(n)$,
	можно решить за $2^{q(n)}q(n)$ на детерминированной МТ, мы использовали
	факт перебора всех возможных отгадок, полиномиальной длины. Смоделируем
	работу такой детерминированной МТ следующим образом: в МТ входит
	генератор всех слов в алфавите А, длина которых не превосходит  $p(n)$
	(таких слов $|A|^p(n)$,
	также входит проверка слова на соответствие определённым требованиям
	(предварительный синтаксический анализ), после генерации и проверки
	работает алгоритм исходной НМТ, проверяющий сгенерированное слово. 

	После проверки слово стирается и на его место записывается новое
	слово-отгадка. Такая МТ работает детерменированно, за экспоненциальное
	время, но с полиномиальным ограничением памяти.
\end{proof}

\begin{proposition}
	$Co-NP \subseteq PSPACE$
\end{proposition}

\begin{theorem}
	$PSPACE \subseteq EXPTIME$
\end{theorem}
\begin{proof}
	Вспомним доказательство Теоремы Кука (\refthm{Cook}), в которой мы для индивидуальной
	задачи задачи $Z\in PSPACE$ строим таблицу выполнимости. Количество
	ячеек в строке Машины Тьюринга ограничена полиномом (без потери общности
	$n^k$ ).
	
	Пусть А --- Алфавит, S --- множество состояний,  $|A\cup S| = m$,
	получается число слов длины $n^k$ ---  $m^{n^k}$.

	В таблице выполнимости нет двух повторяющихся строк, так как иначе
	Машина Тьюринга бы не останавливалась и таблица выполнимости не решала
	задачу. Получается, всего в таблице выполнимости строк  $\le m^{n^k}$.
	Так как число строк определяет время работы Машины Тьюринга, то  $Z\in
	EXPTIME$
\end{proof}

\marginnote[0.5cm]{Сводимость по времени}
\begin{definition}
Задача $Z$ называется $PSPACE$-полной, если  $\forall Z' \in PSPACE: Z' \le_P Z$
\end{definition}

\section{Игра двух лиц}
Рассмотрим множество индивидуальных задач I некоторой массоввой задачи Z. Пусть
есть два игрока --- белый (W) и чёрный (B), которые делают ходы поочереди. Число
ходов заранее ограничено, ходы белого обозначаем $w_1, w_2, \ldots$, чёрного ---
$b_1, b_2, \ldots$. 

Выигрыш будем определять по предикату $W(I, w_1, b_1,
\ldots)$. В силу отсутствия ничьих, истинность предиката будет означать победу
первого игрока, ложность --- второго. При этом победа белого не должна зависеть
от ходов чёрного, то есть предикат можно записать в следующей форме $\exists w_1
\forall b_1 \ldots \exists w_m \forall b_m W(I, w_1,
b_1,\ldots, w_m, b_m)$.

$L_W$ --- множество слов I, на которых выигрывают белые, $L_B$ --- чёрные. 
\begin{theorem}
\labthm{pspace-game}
Задача Z лежит в классе PSPACE тогда и только тогда, когда существует игра с
полиномиальным от длины входа числом ходов и полиномиально вычислимым
результатом такая, что $L_z=L_w$	
\end{theorem}
\begin{proof}
	Пусть число ходов ограничено q(n), А --- алфавит задачи Z, $|A| = k$.
	
	Сначала докажем $\impliedby$, то есть что игра лежит в классе PSPACE.
	Построим корневое дерево T (пример представлен на рисунке \ref{fig:game}), его
	корнем будет I\sidenote{условие игры, вход задачи} затем вершины каждого яруса отражают все возможные ходы первого и
	второго игроков поочерёдно. Каждой вершине можно приписать метку w или
	b.

\begin{figure*}[h!]
    \centering
    \incfig{game}
    \caption{Пример графа, отражающего игру двух лиц}
    \label{fig:game}
\end{figure*}
	Метки будут приписываться следующим образом: для самого нижнего яруса
	$q(n)$ хода чёрного значение однозначно задаётся предикатом  $W(I, w_1,
	b_1, \ldots, w_{q(n)}, b_{q(n)})$. Дальше значения будут задаваться
	рекурсивно, когда мы будем подниматься на ярус выше: 
	\begin{itemize}
		\item если ярус отражает
	ход белого, то он принимает значение w, если хотя бы один из детей
	(ходов) имеет пометку w, иначе ставим b;
		\item если ярус отражает ход чёрного, то он принимает значение
			w, если все дети (ходы) имеют пометку w.
	\end{itemize}

	Такой рекурсией мы однозначно получим значение у I, то есть восстановим
	исход игры на входе I. 

	Значит, для решения задачи Z нужно построить МТ, которая будет вычислять
	результат игры, воспроизводя описанный выше процесс. Число ярусов в
	дереве полиномиально, время проверки значения предиката также
	полиномиально $p(n)$, значит все слова имеют полиномиальную длину. После
	проверки мы будем их стирать. Получается, Машина Тьюринга имеет
	полиномиальную память $O(p(n)q(n))$.

	$(\implies)$. Пусть есть задача Z в классе PSPACE, покажем, что
	существует требуемая игра.

	У нас есть МТ, которая с полиномиальной памятью распознаёт вхождение
	языка в $L_Z$, пусть используемый размер памяти  не превышает $n^k$, построим
	таблицу выполнимости такой машины. Если Машина Тьюринга работает на
	алфавите А с множеством состояний S,  $|A\cup S| = m$, то строк в
	таблице не более $(m)^{n^k}$. Для простоты мажорируем это число при
	помощи $ 2^q$

	Суть игры будет состоять в следующем: белые утверждают, что для I ответ
	да, чёрные хотят это проверить. Понятно, что если для I ответ да, то
	таблица выполнимости будет существовать и будет без ошибок, если нет, то
	где-то в таблице выполнимости будет совершён недопустимый переход. 

	Белые своим ходом выписывают строку 2^{q-1}, чёрные выбирают, в какую из
	двух половин они двинутся, белые выписывают середину выбранного
	диапазона, чёрные снова выбирают. В какой-то момент диапазон будет
	состоять из двух подряд идущих строк. Если состояние одной строки можно
	получить из другой при помощи корректного перехода Машины Тьюринга (они
	отличаются только окошком $2\times 3$, то значит таблица выполнимости
	корректна, белые победили, если нет, то таблица выполнимости
	некорректна, а значит для I ответ нет и победили чёрные. Так как и белые
	и чёрные играют по выигрышной стратегии, то в случае наличия
	некорректности мы к ней придём. 
\end{proof}

 \begin{theorem}
	Задача проверки формулы исчисления предикатов на выполнимость
	PSPACE-полная.
\end{theorem}
 \begin{proof}
	 Любой задаче из PSPACE мы ставим в соответствие игру двух лиц
	 (\refthm{pspace-game}), ответом да или нет на которую определяется по
	 выполнимости соответствующего предиката.
\end{proof}

\section{Отношение PSPACE и NPSPACE}
\begin{theorem}
\labthm{npspace-game}
Задача Z лежит в классе $NPSPACE$ тогда и только тогда, когда существует игра с
полиномиальным от длины входа числом ходов и полиномиально вычислимым
результатом такая, что $L_z=L_w$	
\end{theorem}
\begin{proof}
	$(\impliedby)$. Если существует требуемая игра, то по
	\refthm{pspace-game}  $Z \in PSPACE \implies Z \in NPSPACE$. 

	$(\implies)$. Если $Z\in NPSPACE$, то рассмотрим игру аналогичную
	доказательству в \refthm{pspace-game} только теперь таблица выполнимости
	будет содержать помимо I ещё и отгадку. От увеличения длины
	рассматриваемых строк на полином используемая память также останется
	полиномиальной. 
\end{proof}
\begin{theorem}
\[PSPACE = NPSPACE\]
\end{theorem}
\begin{proof}
	Из \refthm{pspace-game} и \refthm{npspace-game} получаем в силу
	транзитивности, что  $PSPACE = NPSPACE$.
\end{proof}

\begin{theorem}[без доказательства]
	\[NSPACE(f(n)) \le SPACE(f^2(n))\]
\end{theorem}
